%!TEX program = xelatex
\documentclass{article}
\usepackage[top=0.5in, bottom=1in, left=0.75in, right=0.75in]{geometry} % 页边距
\usepackage{xeCJK} % xeCJK 中文最小模板 via http://seisman.info/mini-template-for-xeCJK.html
\usepackage{amsfonts} % \mathbb
\usepackage{amsmath} % \DeclareMathOperator
\usepackage{ebproof}   % Sequent calculi, easier % Thank you, Yinqiu Zhu!
\usepackage{tabularx} % 临时分栏 http://bbs.ctex.org/forum.php?mod=viewthread&tid=78284
\usepackage{array} % 表格 https://zhuanlan.zhihu.com/p/19749566


% 从 hw5 copy
\def\To{\Rightarrow}

\def\a{\alpha}
\def\b{\beta}
\def\c{\gamma}
\def\d{\delta}
\def\C{\Gamma}
\def\D{\Delta}


\def\S{\Sigma}
\def\P{\Pi}

\def\e{\eta}
\def\t{\theta}
\def\T{\Theta}

\def\AX{(AX)}
\def\LB{(L$\bot$)}
\def\LA{(L$\land$)}
\def\LAL{(L$\land_L$)}
\def\LAR{(L$\land_R$)}
\def\RA{(R$\land$)}
\def\LV{(L$\lor$)}
\def\RV{(R$\lor$)}
\def\RVL{(R$\lor_L$)}
\def\RVR{(R$\lor_R$)}

\def\RI{(R$\to$)}
\def\LI{(L$\to$)}

\def\LW{(LW)}
\def\RW{(RW)}
\def\LC{(LC)}
\def\RC{(RC)}

% RC = R Conjunction = R Contraction hmmmmm
% So, RA, RV 2333333
% 重定义宏,没有error没有warning…… 后一个默默覆盖。

% 蛤!TeXstudio 的编辑器 也 可以定义宏…… shift+F1出个证明环境233

\def\EQ{($\equiv$)}

\def\Gonecp{$ \mathsf{G1cp} $ }
\def\Gthreecp{$ \mathsf{G3cp}$ }
\def\Gthreeip{$ \mathsf{G3ip}$ }
\def\Gthreei{$ \mathsf{G3i}$ }
\def\Gonec{$ \mathsf{G1c} $ }
\def\Gonesfour{$ \mathsf{G1S4} $ } % TODO 为啥最后得加空格?

\def\Goneip{$ \mathsf{G1ip} $ }
\def\Gonei{$ \mathsf{G1i} $ }

\def\Gtwocps{$ \mathsf{G2cp^*} $  }
\def\Gtwomips{$ \mathsf{G2[mi]p^*} $  }
\def\Gtwoips{$ \mathsf{G2ip^*} $  }



\def\LVM{(L$\lor^\otimes$)}

\usepackage{amssymb}

\def\Dia{\Diamond} % diamond is small, Diamond is thin

\def\KBox{(K$\Box$)}
\def\RBox{(R$\Box$)}
\def\FBox{(4$\Box$)}
\def\LBox{(L$\Box$)} % LB = L Bottom... 

\def\LDia{(L$\Dia$)}

\def\LN{(L$\neg$)}
\def\RN{(R$\neg$)}
\def\LNN{(L$\neg\neg$)}
\def\RNN{(R$\neg\neg$)}

\def\LAll{(L$ \forall $)}
\def\LExi{(L$ \exists $)}
\def\RAll{(R$ \forall $)}
\def\RExi{(R$ \exists $)}

%\Gonecp
%
%\AX \LB \LAL \LAR \RA \LV \RVL \RVR \RI \LI
%
%\LW \RW \LC \RC
%
%\EQ

\title{证明直觉主义命题逻辑不能导出 Peirce 律和排中律 (精简版)\\(Work in Progress)}
\author{SOnion}

\begin{document}

\maketitle


%This text proves that \Gthreeip, a sequent calculus system for intuitionistic propositional logic, can not derive sequents $~\To((\a\to\b)\to\a)\to\a$ and $~\To \a\lor\neg\a$, thus proving $((\a\to\b)\to\a)\to\a$ (Peirce's law) and $\a\lor\neg\a$ (law of excluded middle) are not valid formulas in intuitionistic propositional logic.

%Peirce 律和排中律是两个重要的命题逻辑公式. Peirce 律是 $((\a\to\b)\to\a)\to\a$, 排中律是 $\a\lor\neg\a$, 其中 $\a,\b$ 是\underline{命题逻辑公式} (下文简称``公式'', 因为本文只在命题逻辑范围内讨论). 

(标题中 ``精简版'' 的意思是, 假设读者对矢列演算 (sequent calculus, 又译\underline{相继式演算}等) 有一定了解, 进而文中不做过多解释; 可能会写一个扩展版, 以此为切入点来介绍矢列演算……)\\

本文通过在直觉主义命题逻辑的一个矢列演算系统 \Gthreeip 中进行反向证明搜索, 证明了矢列 $~\To((\a\to\b)\to\a)\to\a$ 和 $~\To \a\lor\neg\a$ 不可在 \Gthreeip 被导出, 即证明了 $((\a\to\b)\to\a)\to\a$ (Peirce 律) 和 $\a\lor\neg\a$ (排中律) 不是直觉主义命题逻辑的有效式. 

\Gthreeip 是一个\underline{表达}(? ``形式化了''?用啥词好呢)了直觉主义命题逻辑的矢列演算系统, 它由直觉主义一阶逻辑的一个矢列演算系统 \Gthreei 删去四个关于量词的规则而得到(\Gthreei 于 1996 年由 Troelstra 和 Schwichtenberg 在《Basic Proof Theory》中引入). \Gthreeip 的规则如下:


% 临时分栏 http://bbs.ctex.org/forum.php?mod=viewthread&tid=78284
\begin{trivlist}
	\item
	\abovedisplayskip=-\baselineskip
	\belowdisplayskip=-\baselineskip
	\begin{tabularx}{\textwidth}{X|X}
		$$
		\begin{array}{cl}
		
		\displaystyle{\frac{} {p,\C \To p}} & \textmd{\AX} \\ \\
		\displaystyle{\frac{\a,\b,\C \To \d} {\a\land\b,\C \To \d}} & \textmd{\LA} \\ \\
		\displaystyle{\frac{\a,\C \To \d \quad\quad \b,\C \To \d} {\a\lor\b,\C \To \d}} & \textmd{\LV} \\ \\
		\displaystyle{\frac{\a\to\b,\C \To \a \quad\quad \b,\C \To \d} {\a\to\b,\C \To \d}} & \textmd{\LI} \\ \\
	
		\end{array}
		$$
		&
		$$
		\begin{array}{cl}
		
		\displaystyle{\frac{} {\bot,\C \To \a}} & \textmd{\LB} \\ \\
		\displaystyle{\frac{\C \To \a \quad\quad \C \To \b} {\C \To \a\land\b}} & \textmd{\RA} \\ \\
		\displaystyle{\frac{\C \To \a} {\C \To \a\lor\b}} \textmd{\RVL} & \displaystyle{\frac{\C \To \b} {\C \To \a\lor\b}} \textmd{\RVR} \\ \\
		\displaystyle{\frac{\a,\C \To \b} {\C \To \a\to\b}} & \textmd{\RI} \\ \\
		
		\end{array}
		$$
		
	\end{tabularx}
\end{trivlist}

其中 $\a,\b,\d$ 是公式, $p$ 是原子命题, $\C$ 是公式的多重集; $\bot$ 定义成一个零元联结词, 对任意公式 $\a$ 将 $\neg\a$ 定义为 $\a\to\bot$. 值得注意的是, \Gthreeip 中矢列的前件和后件, 都是公式的多重集, 而不是序列或集合 (作为对比, Gentzen 最初提出的矢列演算系统 LK 和 LJ 里面的矢列的前件和后件都是序列, 因此需要 permutation 规则来调整公式的顺序).  

由于反向证明搜索从结论出发, 为了方便表述搜索过程, 本文倒着写证明树, 即: 树根 (待证矢列) 写在最上方; 规则的应用中, 结论写在上方, 前提写在下方.

\section*{Peirce 律不可导出}
从待证矢列 $~ \To ((\a\to\b)\to\a)\to\a$ 出发开始搜索. 在搜索的前几步, 可使用的规则是唯一的, 直到遇到证明枝 (1):\\

% 似乎 & 可以对齐耶
\begin{prooftree}[proof style=downwards]
	\Hypo{}
	\Infer1[\AX]{\a \To \a}
	\Hypo{}
%	\Infer1[\RI~or \LI]{(\a\to\b)\to\a &\To \a\to\b}
	\Infer1[(1)]{(\a\to\b)\to\a &\To \a\to\b}
	\Infer2[\LI]{(\a\to\b)\to\a &\To \a}
	\Infer1[\RI]{~ &\To ((\a\to\b)\to\a)\to\a}
\end{prooftree}\\

未封闭的证明枝 (1) 有两种可行选择: 使用规则 \RI 或 \LI. 先尝试 \RI:\\%(为减少分叉, 不妨总是先尝试 \RI)

\begin{prooftree}[proof style=downwards]
	\Hypo{(\a\to\b)\to\a, \a\to\b, \aa &\To \b}
	\Hypo{}
	\Infer1[(2)]{\a,\a &\To \b}
	\Infer2[\LI]{\a, (\a\to\b)\to\a &\To \b}
	\Infer1[\RI]{(\a\to\b)\to\a &\To \a\to\b}
\end{prooftree}\\

每一步的规则选择都是唯一的, 由此必然产生的未封闭的证明枝 (2), 而 (2) 处已经无规则可用, 因此这个推导路线不可行. 回溯到 (1), 既然此处使用 \RI 不可行, 那么只能选择 \LI:\\

\begin{prooftree}[proof style=downwards]
	\Hypo{\b \To \a\to\b}
	\Hypo{}
	\Infer1[(3)]{(\a\to\b)\to\a &\To \a\to\b}
	\Infer2[\LI]{(\a\to\b)\to\a &\To \a\to\b}
\end{prooftree}\\

使用 \LI 立即产生了证明枝 (3), 而 (3) 的矢列与 (1) 的一样, 产生循环, 因此这个推导路线也不可行. 事实上 \LI 规则中一旦 $\a=\d$, 就会产生循环. 

所以任何一个推导路线都不可行, 即 \Gthreeip $\not\vdash~ \To ((\a\to\b)\to\a)\to\a$, 所以 Peirce 律 $((\a\to\b)\to\a)\to\a$ 在直觉主义命题逻辑中不是有效式.

\section*{排中律不可导出}
$\a\lor\neg\a$ 是 $\a\lor(\a\to\bot)$ 的缩写, 由此开始反向证明搜索. 主联结词是 $\lor$, 因此可以使用 \RVL 或 \RVR. 先尝试 \RVL:\\

\begin{prooftree}[proof style=downwards]
	\Hypo{}
	\Infer1[(1)]{\To \a}
	\Infer1[\RVL]{\To \a\lor(\a\to\bot)}
\end{prooftree}\\

在证明枝 (1) 处无规则可用, 回溯到开头尝试 \RVR:\\

\begin{prooftree}[proof style=downwards]
	\Hypo{}
	\Infer1[(2)]{\a \To \bot}
	\Infer1[\RI]{\To \a\to\bot}
	\Infer1[\RVR]{\To \a\lor(\a\to\bot)}
\end{prooftree}\\

在证明枝 (2) 处无规则可用. 所有推导路线都不可行, 故排中律 $\a\lor\neg\a$ 在直觉主义命题逻辑中不是有效式.

\section*{附: 本文使用的术语}

我尽量用见过的翻译; 有些词我不确定中文里叫啥, 有些我不确定英文里叫啥; 有的词我不知道是否被广泛地(按照我理解的含义)被使用——主要是从老师那里听到的词.

\begin{enumerate}
	\item sequent calculus - 矢列演算 (又译相继式演算等)
	\item antecedent / succedent - 前件 / 后件 (矢列的分隔符(有用$\To$的, 有用$\vdash$的)的左边和右边)
	\item premise / conclusion -  前提 / 结论 (规则的大横线的上边和下边)
	\item proof system, proof calculus - 证明系统 (如公理系统, 各种矢列演算, 各种自然演绎)
	\item derive, deduce - 导出, 推导 (使用证明系统的规则来推导内定理, 如在 \Gthreei 中推导 $\a\To\a$)
	\item (backward) proof search - (反向) 证明搜索 
	\item sequence / multiset / set - 序列 / 多重集 / 集合
	\item ? - 内定理
	\item ? - 证明枝
\end{enumerate}


\section*{参考}


\begin{enumerate}
	\item 俞珺(⿰王君)华老师的证明论课程
	\item Basic Proof Theory 2nd Edition
	\item A terminating evaluation-driven variant of \Gthreei  https://homes.di.unimi.it/fiorentini/download/slidesTableaux2013.pdf
\end{enumerate}



\end{document}
