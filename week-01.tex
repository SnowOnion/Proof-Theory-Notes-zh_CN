%!TEX program = xelatex
\documentclass{article}
\usepackage[top=0.5in, bottom=1in, left=1.25in, right=1.25in]{geometry} % 页边距
\usepackage{xeCJK} % xeCJK 中文最小模板 via http://seisman.info/mini-template-for-xeCJK.html
\usepackage{amsfonts} % \mathbb
\usepackage{amsmath} % \DeclareMathOperator
\usepackage{ebproof}   % Sequent calculi, easier % Thank you, Yinqiu Zhu!
\usepackage{tabularx} % 临时分栏 http://bbs.ctex.org/forum.php?mod=viewthread&tid=78284
\usepackage{array} % 表格 https://zhuanlan.zhihu.com/p/19749566
\usepackage{datetime2} % 不仅要日期还要时间 https://tex.stackexchange.com/questions/2760/omitting-the-date-when-using-maketitle
\usepackage{enumitem} % 定制列表 https://www.latex-tutorial.com/tutorials/lists/

% 2018-09-23 LGC
%\usepackage{mathabx} % \Dashv not beautiful! No double direction!
\usepackage{turnstile} % well… 支持\not(?), 不错啊
% http://mirrors.huaweicloud.com/repository/toolkit/CTAN/macros/latex/contrib/turnstile/turnstile-en/turnstile_article.pdf
% https://ctan.org/pkg/turnstile
\def\seq{\sdststile{}{}} % “semantically equivalent”
\def\nseq{\not\sdststile{}{}}

\usepackage{amsthm} % \begin{proof} 证明过程 \end{proof}
\usepackage{mathrsfs}% \mathscr
\usepackage{graphicx} 
%\includegraphics[scale=0.6]{fullscreen.png}
% eg: width=3cm[缩放因子], height=8 cm[缩放因子] scale=0.4[缩放因子]
% 浮动体

%\begin{figure}[ht]
%	\centering
%	\includegraphics[width=13.5cm]{2(b).JPG}
%	\caption{Expected Truth Tables of $I_0,\cdots,I_4$ }
%	\label{fig:proving-imply-cannot-derive-and}
%\end{figure}


% 从 PT 2017 hw5 copy 改造(该模块化了!)

\def\To{\Rightarrow}

\def\a{\alpha}
\def\b{\beta}
\def\c{\gamma}
\def\d{\delta}
\def\C{\Gamma}
\def\D{\Delta}


\def\S{\Sigma}
\def\P{\Pi}

\def\e{\eta}
\def\t{\theta}
\def\T{\Theta}

\def\AX{(AX)}
\def\LB{(L$\bot$)}
\def\LA{(L$\land$)}
\def\LAL{(L$\land_L$)}
\def\LAR{(L$\land_R$)}
\def\RA{(R$\land$)}
\def\LV{(L$\lor$)}
\def\RV{(R$\lor$)}
\def\RVL{(R$\lor_L$)}
\def\RVR{(R$\lor_R$)}

\def\RI{(R$\to$)}
\def\LI{(L$\to$)}

\def\LW{(LW)}
\def\RW{(RW)}
\def\LC{(LC)}
\def\RC{(RC)}

% RC = R Conjunction = R Contraction hmmmmm
% So, RA, RV 2333333
% \def 重定义宏, 没有error没有warning…… 后一个默默覆盖。\newcommand 则会检查, 不能覆盖已有的命令。\renewcommand 则又能覆盖……
% 怎么快速找到哪个命令在哪个宏包被定义的呢?

% 蛤!TeXstudio 的编辑器 也 可以定义宏(快捷键层面)…… shift+F1出个证明环境233

\def\EQ{($\equiv$)}

\def\Gonecp{$ \mathsf{G1cp} $ }
\def\Gthreecp{$ \mathsf{G3cp}$ }
\def\Gthreeip{$ \mathsf{G3ip}$ }
\def\Gthreei{$ \mathsf{G3i}$ }
\def\Gonec{$ \mathsf{G1c} $ }
\def\Gonesfour{$ \mathsf{G1S4} $ } % TODO 为啥最后得加空格?

\def\Goneip{$ \mathsf{G1ip} $ }
\def\Gonei{$ \mathsf{G1i} $ }

\def\Gtwocps{$ \mathsf{G2cp^*} $  }
\def\Gtwomips{$ \mathsf{G2[mi]p^*} $  }
\def\Gtwoips{$ \mathsf{G2ip^*} $  }



\def\LVM{(L$\lor^\otimes$)}

\usepackage{amssymb}

\def\Dia{\Diamond} % diamond is small, Diamond is thin

\def\KBox{(K$\Box$)}
\def\RBox{(R$\Box$)}
\def\FBox{(4$\Box$)}
\def\LBox{(L$\Box$)} % LB = L Bottom... 

\def\LDia{(L$\Dia$)}

\def\LN{(L$\neg$)}
\def\RN{(R$\neg$)}
\def\LNN{(L$\neg\neg$)}
\def\RNN{(R$\neg\neg$)}

\def\LAll{(L$ \forall $)}
\def\LExi{(L$ \exists $)}
\def\RAll{(R$ \forall $)}
\def\RExi{(R$ \exists $)}

%\Gonecp
%
%\AX \LB \LAL \LAR \RA \LV \RVL \RVR \RI \LI
%
%\LW \RW \LC \RC
%
%\EQ

\title{Junhua Yu 证明论 2017 - SOnion 笔记 (草稿)}
\author{snowonionlee@gmail.com}
\date{\DTMnow}

\begin{document}

\maketitle

本文期待的读者是啥样的人呢? 如果您完全不懂符号逻辑或者叫数理逻辑或简称逻辑, 肯定没法以本文来入门; 如果您学过一些逻辑, 可能能从本文得到各个逻辑的关系的梳理---也很可能从本文找出大大小小错误来, 不过我保证公开发布 (QQ 和微信群 = 不太公开, GitHub = 公开) 前我至少会自己再三审读.

形式语言很重要, 解读它们的自然语言也很重要.

时间所限, 先只写核心的东西, 所以基本上写的都是形式语言, 起承转合的自然语言很少.

我 \LaTeX 不熟, 排版中文更不熟, 所以就先看个影儿吧.

\section{当我们谈论``一个逻辑''、``一个证明系统''的时候在谈论什么}

\subsection{逻辑}

我们在此约定: 本笔记系列谈论``一个逻辑''的时候, 说的是这个集合的一个元素:
%$$
%\{\text{极小逻辑, 直觉主义逻辑, 古典逻辑,} \cdots\} \times \{命题逻辑, 一阶逻辑, 模态逻辑, \cdots\}
%$$
$$
\{\text{极小逻辑}, \text{直觉主义逻辑}, \text{古典逻辑}, \cdots\} \times \{\text{命题逻辑}, \text{一阶逻辑}, \text{模态逻辑}, \cdots\}
$$

譬如``古典命题逻辑''是一个逻辑、``直觉主义模态逻辑''是另一个逻辑. 
% 选择一组逻辑联结词, 选定一种具体的形式文法, 

同一个逻辑不一定总是``长得一样''. 有两个层次能让同一个逻辑的``外貌''发生变化. 以古典命题逻辑为例:

第一个层次, 选择不同的逻辑联结词集合, 该逻辑的公式的 (作为树的) 形态会不一样. 本笔记系列选取的联结词有 $\to,\land,\lor,\bot$. 其中, 蕴含、合取、析取是二元的, bottom 是零元的.

第二个层次, 对给定的逻辑联结词集合, 选择不同的具体语法, 该逻辑的公式的 (作为字符串的) 形态会不一样. 我们选取这个具体语法:

$$
\phi ::= A \mid \bot \mid (\phi \to \phi) \mid (\phi \land \phi) \mid (\phi \lor \phi)
$$ 

其中 $A$ 代表原子命题. 我们常使用小写拉丁字母表示原子命题, 使用小写希腊字母表示公式.

并作以下约定: 
\begin{enumerate}
\item 公式最外层的括号可以省略;
\item 优先级方面, 合取 > 析取 > 蕴含, 在此基础上若不引起歧义可以省略括号;
\item 结合性, 合取、析取、蕴含规定为右结合, 在此基础上若不引起歧义可以省略括号.
\end{enumerate}
如: 把 $a \to b \to a$ 补全括号后是 $(a \to (b \to a))$, 把 
$p \land q \land r \lor u \lor v \to w$ 补全括号之后是 
$(((p \land (q \land r)) \lor (u \lor v)) \to w)$.

这样一来, 原子命题 $p$ 是公式, $p \to \bot$ 是公式, $p \to$ 不是公式 (不符合具体语法), $\neg p$ 不是公式 (我们没有选取 $\neg$ 作为逻辑联结词!).

当我们需要引入``否定''的时候, 会把 $\neg \a$ 仅仅当成一个语法上的缩写, 将 $\neg \a$ 定义为 $\a \to \bot$. 也就是说, 写下 $\neg \a$ 的效果等同于写下 $\a \to \bot$ (并补充必要的括号). 值得指出, 在此种定义之下, $\neg$ 是元语言的符号, 而不是我们讨论的对象语言---命题逻辑的语言---的符号. 我们将会说 $\neg p$ 是公式---这其实是在说 $p \to \bot$ 是公式.\\

注: 
\begin{itemize}
	\item 事实上, 这个语法不仅适用于古典命题逻辑, 还适用于极小命题逻辑和直觉主义命题逻辑.
	\item 如果您有一些计算机科学, 特别是程序语言的知识, 可以看出: 第一个层次是抽象语法, 第二个层次是具体语法.
\end{itemize}


\subsection{证明系统}

读者大概已经知道, 讨论一个逻辑, 有两个截然不同的方面: 语义的和语形的. ``真、假、赋值、解释、模型、永真式 (有效式)、可满足、不可满足、$\models$、$\Vdash$'' 这些词汇和符号归属语义那边,``推导、公理、推理规则、定理 (内定理)、可证明、不可证明、$\vdash$'' 则归属语形这边. 这是结构证明论的笔记, 因此讨论的几乎全都是语形的方面.

``一个证明系统''关心一个逻辑的语形推导. 一个证明系统的``类型''**是这个集合的一个元素:
$$
\{\text{极小逻辑}, \text{直觉主义逻辑}, \text{古典逻辑*}, \cdots\} \times \{\text{命题逻辑}, \text{一阶逻辑}, \text{模态逻辑}, \cdots\} \times \{\text{公理系统}, \text{矢列演算}, \text{自然演绎}, \cdots\}
$$

极小、直觉主义、古典是逻辑的最常见的三种基底; 公理系统、矢列演算和自然演绎则是最常见的三类证明系统框架, 《Basic Proof Theory 第二版》称之为``three types of formalism''.

譬如我们在矢列演算的章节要详细谈的 \Gonecp, 就是古典命题逻辑的矢列演算系统的一种. c for classical (古典), p for propositional (命题), G for Gentzen (根岑, 矢列演算的发明人), 1 是个``编号''---古典命题逻辑的矢列演算系统不止一种.\\

注:
\begin{itemize}
	\item * 在本笔记系列中, ``古典逻辑''和``经典逻辑''是同义词, ``一阶逻辑''和``一阶谓词逻辑''是同义词.
	\item ** 此处``类型''不是严格的术语!
\end{itemize}

我们紧接着要详细谈的是命题逻辑的公理系统.


\section{希尔伯特风格的公理系统}

在本笔记系列中, ``公理系统''、``希尔伯特风格的公理系统''、``希尔伯特风格系统''是同义词.

公理系统这种证明系统的特点是, 公理多, 规则少 (只有 MP 规则). 相比之下, 矢列演算的特点是公理少 (0 个), 规则多.

须知, 选取不同的公理集合, 也可能证明出同样的定理集合---也就是说形式化了同一个逻辑. 下文我们选取的是其中一种.

极小命题逻辑、直觉主义命题逻辑、古典命题逻辑中都只有 MP 规则, 而公理依次增多. 直觉主义是极小加上``bottom 蕴含一切'', 古典是直觉主义加上双否消去或者加上排中律. 我们统一叙述 MP 规则, 然后依次叙述它们的公理.

MP 规则为:

\begin{prooftree} 
	\Hypo{\C \vdash \a}
	\Hypo{\D \vdash \a \to \b}
	\Infer2[(MP)]{\C \cup \D \vdash \b} 
\end{prooftree}	\\

其中 $\a, \b$ 为公式, $\C, \D$ 为公式的集合, $\cup$ 是集合的并运算. MP 规则的含义为: 如果以 $\C$ 为前提能推导出 $\a$, 以 $\D$ 为前提能推导出 $\a \to \b$, 那么从公共的前提 $\C \cup \D$ 就能推导出 $\b$.

是不是比 
\begin{prooftree} 
	\Hypo{\a}
	\Hypo{\a \to \b}
	\Infer2[]{\b} 
\end{prooftree}	
要酷炫一些? :) 

因为通用所以酷炫.

\subsection{极小命题逻辑}

采用 1.1 中定义的命题逻辑语言. 

公理模式有 8 条:

\begin{enumerate}
\item $\vdash \a \to (\b \to \a)$
\item $\vdash (\a \to (\b \to \a)) \to ((\a \to \b) \to (\a \to \c))$
\item $\vdash \a \to \a \lor \b$
\item $\vdash \b \to \a \lor \b$
\item $\vdash (\a \to \c) \to ((\b \to \c) \to ((\a \lor \b) \to \c))$
\item $\vdash \a \land \b \to \a$
\item $\vdash \a \land \b \to \b$
\item $\vdash \a \to (\b \to (\a \land \b))$
\end{enumerate}

现在可以说一些注意事项了:

\begin{enumerate}
\item 注意到公理模式里没有 $\bot$ (因此也没有 $\neg$). 这导致, 极小命题逻辑里 $\bot$ 没有任何特殊性, 就是一个普通原子命题.
\item 什么是``公理模式 (axoim schema)''? 因为其中的 $\a, \b, \c$ 都是元语言符号 (更具体地, ``元语言变元''), 将公理模式里的它们替换成任意公式, 都能得到一个具体的公理. 这样子, 8 条公理模式对应着可数无穷条公理. 类似地, 我们也说 ``定理模式''.
\item 为什么一直说``推导''而避免说``证明''呢? 我用``推导''表示: 从公理和一些公式前提出发, 应用推理规则, 得到一个公式作为结论. Reserve ``prove'' in meta level, 即``证明''我一般指元定理 (譬如演绎定理、可靠性、完全性) 的证明.
\item 公理 3 和 4 是否重复? 不. 我们讨论语形推导, 不涉及语义, 所以 $\a \lor \b$ 和 $\b \lor \a$ 没什么关系.
\end{enumerate}

我们来在极小命题逻辑里推导\emph{同一律} $\vdash \a \to \a$, 顺便展示公理系统的内定理的证明的树形的写法 (而非常见的序列形的写法).
\begin{proof}.\\
\begin{prooftree} 
	\Hypo{}
	\Infer1[(公理1)]{\vdash \a \to ((\a \to \a) \to \a)}
	\Hypo{}
	\Infer1[(公理2)]{\vdash (\a \to ((\a \to \a) \to \a)) \to ((\a \to (\a \to \a)) \to (\a \to \a))}
	\Infer2[(MP)]{\vdash (\a \to (\a \to \a)) \to (\a \to \a)} 
	\Hypo{}
	\Infer1[(公理1)]{\vdash \a \to (\a \to \a)}
	\Infer2[(MP)]{\vdash \a \to \a} 
\end{prooftree}
\end{proof}
(喷了! 突破天际了. 这可咋整……)

注意到我们叙述的同一律也是个定理模式, 而非一个具体的定理.~:D

\subsection{直觉主义命题逻辑}

在极小命题逻辑的基础上, 加一个公理模式, 即得到直觉主义命题逻辑:

\begin{enumerate}[start=9]
\item $\vdash \bot \to \a$
\end{enumerate}

随着公理的增多, 能推导的定理也会变多. 推导同一律只需要公理 1 和公理 2, 所以同一律是极小命题逻辑的内定理, 是直觉主义命题逻辑的内定理, 也是古典命题逻辑的内定理.

如 1.1 所提到的, 对于公式 $\a$, 将 $\neg \a$ 定义为 $\a \to \bot$. 再次强调这是一个纯语法的替换; $\neg$ 是元语言的符号. 规定 $\neg$ 的优先级比 $\land$ 更高.

在直觉主义命题逻辑里不能推导出这两个内定理: $\vdash \neg \neg \a \to \a$ 和 $\vdash \a \lor \neg \a$. ``$\vdash \phi$ 不是形式系统 $H$ 的内定理'', 或者说 ``$H$ 不能推导出公式 $\a$'', 这是关于 $H$ 的一个元定理, 或者说是 $H$ 的一个性质; 它可以记作 $H \not\vdash \a$ (这个符号可能让人不太舒服: 以前 $\vdash$ 左边都是公式集, 现在怎么是一个形式系统? 还好, 一般不会有歧义). ``不能推导出 (underivability)'' (或者为修辞而暂时不严格一下, ``不可证'') 是可以证明的! 虽然, 在公理系统里证明 underivability 比较难, 在矢列演算里则相对简单.

\subsection{古典命题逻辑}

在直觉主义命题逻辑的基础上, 加以下两个公理模式之一, 即得到古典命题逻辑:
\begin{enumerate}[start=10]
\item $\vdash \neg \neg \a \to \a$ (双否消去)
\end{enumerate}
\begin{enumerate}[start=10]
\item $\vdash \a \lor \neg \a$ (排中律)
\end{enumerate}

读者练习 (一个证内定理, 一个证元定理): 

\begin{enumerate}
\item 在古典命题逻辑里推导定理模式 $\vdash (\neg \b \to \neg \a) \to (\a \to \b)$.
\item 记公式模式 $\eta = (((\a \to \b) \to \a) \to \a)$. 请证明或证伪: 直觉主义命题逻辑加上 $\eta$ 作为公理, 能推导出所有古典命题逻辑的定理 (这可以记作 $I_{pc} \oplus \eta \supseteq C_{pc}$).
\end{enumerate}


\section{版权、免责和负责声明}
\begin{itemize}
\item 暂时, 作者 SOnion 保留一切权利. 也许以后 CC 吧.
\item 本文在法律上不对正确性作保证, 引用本文所造成的损失恕不负责.
\item 如果本文有错, 一般是我的理解、记录或打字错误, 而非 Junhua Yu 讲错~:D
\end{itemize}






\end{document}
