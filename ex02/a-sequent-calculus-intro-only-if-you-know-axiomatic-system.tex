%!TEX program = xelatex
% 注意在 Overleaf 要手动把 project 的 compiler 改成 XeLaTeX
\documentclass{article}
\usepackage[top=0.5in, bottom=1in, left=1.25in, right=1.25in]{geometry} % 页边距
\usepackage{xeCJK} % xeCJK 中文最小模板 via http://seisman.info/mini-template-for-xeCJK.html
\usepackage{amsfonts} % \mathbb
\usepackage{amsmath} % \DeclareMathOperator
\usepackage{ebproof}   % Sequent calculi, easier % Thank you, Yinqiu Zhu!
\usepackage{tabularx} % 临时分栏 http://bbs.ctex.org/forum.php?mod=viewthread&tid=78284
\usepackage{array} % 表格 https://zhuanlan.zhihu.com/p/19749566
\usepackage{datetime2} % 不仅要日期还要时间 https://tex.stackexchange.com/questions/2760/omitting-the-date-when-using-maketitle
\usepackage{enumitem} % 定制列表 https://www.latex-tutorial.com/tutorials/lists/

% 2018-09-23 LGC
%\usepackage{mathabx} % \Dashv not beautiful! No double direction!
\usepackage{turnstile} % well… 支持\not(?), 不错啊
% http://mirrors.huaweicloud.com/repository/toolkit/CTAN/macros/latex/contrib/turnstile/turnstile-en/turnstile_article.pdf
% https://ctan.org/pkg/turnstile
\def\seq{\sdststile{}{}} % “semantically equivalent”
\def\nseq{\not\sdststile{}{}}

\usepackage{amsthm} % \begin{proof} 证明过程 \end{proof}
\usepackage{mathrsfs}% \mathscr
\usepackage{graphicx} 
%\includegraphics[scale=0.6]{fullscreen.png}
% eg: width=3cm[缩放因子], height=8 cm[缩放因子] scale=0.4[缩放因子]
% 浮动体

%\begin{figure}[ht]
%	\centering
%	\includegraphics[width=13.5cm]{2(b).JPG}
%	\caption{Expected Truth Tables of $I_0,\cdots,I_4$ }
%	\label{fig:proving-imply-cannot-derive-and}
%\end{figure}


% 从 PT 2017 hw5 copy 改造(该模块化了!)

\def\To{\Rightarrow}

\def\a{\alpha}
\def\b{\beta}
\def\c{\gamma}
\def\d{\delta}
\def\C{\Gamma}
\def\D{\Delta}


\def\S{\Sigma}
\def\P{\Pi}

\def\e{\eta}
\def\t{\theta}
\def\T{\Theta}

\def\AX{(AX)}
\def\LB{(L$\bot$)}
\def\LA{(L$\land$)}
\def\LAL{(L$\land_L$)}
\def\LAR{(L$\land_R$)}
\def\RA{(R$\land$)}
\def\LV{(L$\lor$)}
\def\RV{(R$\lor$)}
\def\RVL{(R$\lor_L$)}
\def\RVR{(R$\lor_R$)}

\def\RI{(R$\to$)}
\def\LI{(L$\to$)}

\def\LW{(LW)}
\def\RW{(RW)}
\def\LC{(LC)}
\def\RC{(RC)}

% RC = R Conjunction = R Contraction hmmmmm
% So, RA, RV 2333333
% \def 重定义宏, 没有error没有warning…… 后一个默默覆盖。\newcommand 则会检查, 不能覆盖已有的命令。\renewcommand 则又能覆盖……
% 怎么快速找到哪个命令在哪个宏包被定义的呢?

% 蛤!TeXstudio 的编辑器 也 可以定义宏(快捷键层面)…… shift+F1出个证明环境233

\def\EQ{($\equiv$)}

\def\Gonecp{$ \mathsf{G1cp} $ }
\def\Gthreecp{$ \mathsf{G3cp}$ }
\def\Gthreeip{$ \mathsf{G3ip}$ }
\def\Gthreei{$ \mathsf{G3i}$ }
\def\Gonec{$ \mathsf{G1c} $ }
\def\Gonesfour{$ \mathsf{G1S4} $ } % TODO 为啥最后得加空格?

\def\Goneip{$ \mathsf{G1ip} $ }
\def\Gonei{$ \mathsf{G1i} $ }

\def\Gtwocps{$ \mathsf{G2cp^*} $  }
\def\Gtwomips{$ \mathsf{G2[mi]p^*} $  }
\def\Gtwoips{$ \mathsf{G2ip^*} $  }



\def\LVM{(L$\lor^\otimes$)}

\usepackage{amssymb}

\def\Dia{\Diamond} % diamond is small, Diamond is thin

\def\KBox{(K$\Box$)}
\def\RBox{(R$\Box$)}
\def\FBox{(4$\Box$)}
\def\LBox{(L$\Box$)} % LB = L Bottom... 

\def\LDia{(L$\Dia$)}

\def\LN{(L$\neg$)}
\def\RN{(R$\neg$)}
\def\LNN{(L$\neg\neg$)}
\def\RNN{(R$\neg\neg$)}

\def\LAll{(L$ \forall $)}
\def\LExi{(L$ \exists $)}
\def\RAll{(R$ \forall $)}
\def\RExi{(R$ \exists $)}

%\Gonecp
%
%\AX \LB \LAL \LAR \RA \LV \RVL \RVR \RI \LI
%
%\LW \RW \LC \RC
%
%\EQ




% 只有本文才用的:
\def\theH{H_{cp\to}}
\def\theG{G1_{cp\to}}


\title{问道于群}
\author{SOnion snowonionlee@gmail.com}
\date{\DTMnow}

\begin{document}
	
	\maketitle
	
	\section{下述命题逻辑的某片段的两个演算等价吗?}
	
	\subsection{语言}
	$$\varphi ::= p ~|~ (\varphi \to \varphi)$$
	
	并按习俗省略括号……
	
	约定:本文中的 $\C$、$\D$ 是多重集, 不是集合或序列. Sequent 用 $\To$ 作为 antecedent 和 succedent 的分隔符. $\vdash$ 只用于在元语言里表示可证性.
	
	\subsection{一个公理系统, 暂且称为 $\theH$}
	.
	
	规则:\\
	
	\begin{prooftree} 
		\Hypo{\C \vdash \a}
		\Hypo{\D \vdash \a \to \b}
		\Infer2[(MP)]{\C \cup \D \vdash \b} 
	\end{prooftree}	\\
	
	公理: 
	
	\begin{itemize}
		\item AX1: $\vdash \a \to (\b \to \a)$
		\item AX2: $\vdash (\a \to (\b \to \c)) \to ((\a \to \b) \to (\a \to \c))$
	\end{itemize}
	
	\subsection{一个矢列演算, 暂且称为 $\theG$}
	.
	
	(从名字可以看出, 是从 \Gonecp 里掐出来的. 我确实是这样干的.)
	
	公理: 无.
	
	规则: 一个公理规则, 两个逻辑规则, 四个结构规则.\\
	
	\begin{prooftree} 
		\Hypo{}
		\Infer1[(AX)]{\a \to \a} 
	\end{prooftree} \\
	
	\begin{prooftree} 
		\Hypo{\a,\C \To \D, \b}
		\Infer1[(R$\to$)]{\C \To \D, \a\to\b} 
	\end{prooftree}
\quad
	\begin{prooftree} 
	\Hypo{\b,\C \To \D}
	\Hypo{\C \To \D,\a}
	\Infer2[(L$\to$)]{\a\to\b, \C \To \D } 
	\end{prooftree} \\

	\begin{prooftree} 
	\Hypo{\C \To \D}
	\Infer1[(LW)]{\a,\C \To \D} 
	\end{prooftree}
\quad
	\begin{prooftree} 
	\Hypo{\C \To \D}
	\Infer1[(RW)]{\C \To \D,\b} 
	\end{prooftree}
\quad
	\begin{prooftree} 
	\Hypo{\a,\a,\C \To \D}
	\Infer1[(LC)]{\a,\C \To \D} 
	\end{prooftree}
\quad
	\begin{prooftree} 
	\Hypo{\C \To \D,\b,\b}
	\Infer1[(RC)]{\C \To \D,\b} 
	\end{prooftree} \\

	
	\subsection{我的疑问}
	\begin{enumerate}
		\item $\theH$ 和 $\theG$ 等价吗? 这里把等价定义为, 对任意公式$\varphi$和公式多重集$\S$有$\S \vdash_{\theH} \varphi$当且仅当$\vdash_{\theG} \S \To \varphi$. 由于这题是我瞎出的, 且我在试图证右边推左边时, 卡在 (L$\to$) 等几个规则过不去, 所以怀疑根本不正确.
		\item $\theH$ 或 $\theG$ 可以叫作 ``古典命题逻辑的蕴含词片段'' 吗?
	\end{enumerate}
	
	
\end{document}
